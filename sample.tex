\documentclass{thesis}

% TX Fonts を使う
\usepackage{txfonts}

\begin{document}

% 表紙
\title{{\bf コンピュータサイエンスに関する研究}}
\author{愛媛大学 大学院理工学研究科\\
電子情報工学専攻 情報工学コース\\
08460001A  愛媛 太郎\\
\ \vspace{0.5cm} \\
指導教員\\
松山 太郎, 愛媛 花子, 東温 次郎}
\date{2026年2月10日提出}
\maketitle

% 目次
\tableofcontents

\chapter{序論}

本研究では情報工学の観点から○○の有効性を検証する。

\section{研究の背景}

いろいろあります\cite{ソフトウェア工学の基礎知識,ファジィ学会誌-解説,Fenton_and_Kaposi_1987}.

\begin{enumerate}[(1)]
\item あれとか
\item これとか
\item それとか
\end{enumerate}

\chapter{いろいろ}

\section{定義}

\begin{definition}[距離] \label{def:距離} ~

\begin{rm}
$2$ つのベクトル $\boldsymbol{x}$, $\boldsymbol{y}$ が与えられたとき,
これらの距離 $d(\boldsymbol{x},\boldsymbol{y})$ を次式で定義する:
%
\begin{equation}\label{eqn:距離}
	d(\boldsymbol{x},\boldsymbol{y}) = %
	|| \boldsymbol{x} - \boldsymbol{y} ||
\end{equation}
\end{rm}
%
\QED
\end{definition}

一般に定義 \ref{def:距離} における式(\ref{eqn:距離})の関数 $d$ は,
距離関数と呼ばれている\footnote{信じないでね.}.
距離関数が定義された空間{\--}正確にいえばベクトル空間{\--}を距離空間とい
う\footnote{これまた,信じないでね}.

\section{図とか}

\begin{figure}[H]
 \center
 \includegraphics[scale=0.3]{./golfer.eps}
 \caption{図の例}
\end{figure}

\chapter{結論}

以下の注意点があります.

\begin{enumerate}[(i)]
\item 文献を\tcite{ソフトウェア工学の基礎知識}のように参照したい場合は,citeの代わりにtciteを使います.
\item 定理環境には,定義definitionの他に,定理theorem,系corollary,補助定理lemma,命題proposition,仮定assumptionが使えます.
\end{enumerate}

\acknowledgement

お礼を書きます.

\begin{thebibliography}{99}
\bibitem{ソフトウェア工学の基礎知識}
白鳥則郎,高橋薫,神長裕明,
``ソフトウェア工学の基礎知識,''
昭晃堂,1997.
%
\bibitem{ファジィ学会誌-解説}
松本健一,
``解説 ソフトウェアメトリクス,'' 
日本ファジィ学会誌,vol.10, no.5, pp.796--803, Oct.\ 1998.
%
\bibitem{Fenton_and_Kaposi_1987}
N.E.\ Fenton and A.A.\ Kaposi, 
``Metrics and software structure,''
Journal of Information and Software Technology,
29, pp.301--320, July 1987.
\end{thebibliography}

\appendix

\chapter{その他}

ここでは番号がすべてアルファベットに変わります.

\end{document}


